% Makes use of `glossaries`

\DeclareDocumentCommand{\newdualentry}{ O{} O{} m m m m } {
	\newglossaryentry{gls-#3}{name={#5},text={#5\glsadd{#3}},
		description={#6},#1
	}
	\makeglossaries
	\newacronym[see={[Glossary:]{gls-#3}},#2]{#3}{#4}{#5\glsadd{gls-#3}}
}
% Syntax: \newdualentry[glossary options][acronym options]{label}{abbrv}{long}{description}


% ========== Acronyms/Abbreviations ===========

\newacronym{gcc}{GCC}{GNU Compiler Collection}

\newacronym{msvc}{MSVC}{Microsoft C/C++}

% =============================================

\makeglossaries

% ============== Glossary Terms ===============

\newglossaryentry{class}{
	name=class,
	description={In object-oriented programming, an abstract representation of a concrete object, in a way
		that can be understood by a computer},
}

\newglossaryentry{object}{
	name=object,
	description={An instance of a class},
	see=[see also]{{instance},{class}},
}

\newglossaryentry{instance}{
	name=instance,
	description={In object-oriented programming, an instance is
		a concrete occurrence of an abstract \gls{class}},
	see=[see also]{class},
}

\newglossaryentry{transcode}{
	name=transcode,
	description={Convert from one form of encoding to another},
}

\newglossaryentry{pipeline}{
	name=pipeline,
	description={In software, a chain of elements to be processed, where
		the output of one element is the input of the next},
}

\newglossaryentry{library}{
	name=library,
	description={In software, a collection of related
		functions that can be utilized in other software},
	plural=libraries,
}

\newglossaryentry{bigo}{
	name=Big O,
	description={A mathematical notation describing the limiting function
		of an operation. Standard notation is $O(x)$, where x is the
		limiting function},
}

\newglossaryentry{rolling}{
	name=rolling,
	description={In statistics, a rolling statistic is a statistic
		calculated over a "window" of set size. As the window size is
		reached old samples are discarded and ignored in favour of new
		samples},
}

\newglossaryentry{tech-debt}{
	name=technical debt,
	description={In software, code that is developed because it was easy
		to implement and the best solution at the time but is no longer the
		best overall solution},
}

\newglossaryentry{work-unit}{
	name=unit of work,
	description={The smallest logical collection of work required to be
		done in a larger process; generally atomic. All portions of the unit
		must complete successfully, or fail completely without side-effects
	\cite{martinfowler2003}},
	plural=units of work,
}

\newglossaryentry{real-time}{
	name=real-time,
	description={A system which processes data in a time frame that is
		virtually immediate},
}

\newglossaryentry{threadsafe}{
	name=thread safe,
	description={Code or an object that must interact with it's data
		structures in such a way that can be executed from multiple
		threads and preserve integrity},
}

\newglossaryentry{mutex}{
	name=mutex,
	description={An object used to enforce \gls{mutual-exclusion}},
	see={mutual-exclusion},
}

\newglossaryentry{mutual-exclusion}{
	name=mutual exclusion,
	description={A control method used in concurrent programming to
		prevent a race condition. It is the requirement that threads never
		enter critical sections simultaneously},
}

\newglossaryentry{thread}{
	name=thread,
	description={The smallest sequence of instructions that can be managed
		by a scheduler},
}

\newglossaryentry{stream}{
	name=stream,
	description={A sequence of data operated with no determinate end,
		must be processed in sections rather than a whole},
}

% ============================================

% =============== Dual Entries ===============

% ============================================

\makeglossaries
