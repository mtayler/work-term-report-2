\section{Summary}

\enlargethispage{2\normalbaselineskip}
A fragmented codebase, caused by differing execution paths and compilation based
on platform where this report is concerned, increases development time and cost,
hurts maintainability, and increases the probability for failures and flaws in code to
occur. Unification of the codebase by writing platform agnostic code decreases
development time and cost, improves maintainability, and reduces the change of
failures and flaws arising. A new \gls{real-time} processing \gls{pipeline} for
\glspl{stream} requiring processing was created with emphasis on cross-platform
support; this new \gls{pipeline} didn't implement metrics capability unlike the
previous fragmented implementation.

In an effort to provide improved debugging support and complete feature coverage
a new metrics framework was designed, created, and integrated with the new
processing \gls{pipeline}. This feature will increase usage of the new
platform-agnostic platform and further the goal of a unified codebase.

Through the design of this metrics framework many challenges were encountered
providing opportunities to use engineering knowledge and experience to solve.
Deciding on the best and most suitable data structures to use, the general
architecture of the framework, how best to integrate the framework with the
existing \gls{pipeline} code, and other issues such as \gls{threadsafe}ty were
all challenges encountered in the development of the metrics framework.  The
final design of the metrics framework was a metrics \gls{object} that stored
samples and provided calculation of the metrics and printing functionality for
"snapshots" of the metrics that were created to enforce consistency between a
request for the calculation of the metrics and their printing; which was then
integrated with the processing \gls{pipeline} through existing callback
functionality. To add samples to the metrics object, the callbacks already
present in the \glspl{work-unit} being processed were used to push the current
time or a time duration referenced from the start of the event of interest,
depending on the metric(s) the sample is used to compute.  To ensure the ability
to process \glspl{work-unit} was minimally affected, any processing was limited
to when the current value of the metrics are retrieved, as values are pushed on
the execution \gls{thread} of the processing \gls{pipeline} and could be
retrieved on a separate \gls{thread}.

The metrics framework brought the new cross-platform \gls{pipeline} to
feature parity with the old platform-specific \gls{pipeline}: furthering
unification of the codebase and providing the benefits that come with
unification of the codebase. It is recommended to keep pursuing this goal of a
fully unified and cross-platform codebase.
